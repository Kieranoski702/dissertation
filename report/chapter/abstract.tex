\chapter*{Abstract}

This dissertation presents a novel system designed to enable remote multiplayer
gameplay on the Nintendo Wii console while preserving the original local
multiplayer experience. Local multiplayer gaming has historically provided
unique communal experiences, but as online gaming has become dominant, these
traditional modes have diminished, leaving a significant portion of the Wii’s
legacy inaccessible. This work introduces a solution that enables remote players
to participate in Wii games originally designed for local play.

The system integrates three core components. A capture and streaming subsystem
is responsible for real-time extraction and transmission of the Wii’s video and
audio outputs. A controller input relay converts Wii Remote signals into
fixed-length binary packets and transmits them over a low-latency network
connection. Finally, a Wii Remote emulator running on the host device interprets
the relayed inputs and communicates with the console via Bluetooth. This
configuration strives to preserve the original responsiveness and interactive
quality of the Wii experience, while addressing challenges such as transmission
delay and input accuracy.

The project addresses key technical obstacles including latency minimisation,
accurate input replication, and audiovisual quality maintenance. Rigorous
testing and iterative design confirm that remote gameplay can closely
approximate native local multiplayer interactions, despite residual challenges.

\begin{center}
	\noindent\rule{8cm}{0.4pt}
\end{center}
