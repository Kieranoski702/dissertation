\chapter{Conclusion}
\label{chapter:conclusion}

This dissertation set out to explore the feasibility of remote multiplayer
gameplay on the Nintendo Wii console while retaining local multiplayer dynamics.
Through the development of a comprehensive system that integrates audiovisual
streaming, low-latency input relay, and Wii Remote emulation, the project has
successfully demonstrated the feasibility of remote gameplay that closely mimics
the native local multiplayer experience.

The project’s most significant achievement lies in its successful integration of
multiple complex components into a cohesive system. The implementation of a
custom input relay, in tandem with a modified Wii Remote emulator, allowed for
real-time transmission of controller inputs. This solution ensured that remote
players could experience accurate and responsive control, even though some
latency remained compared to native gameplay.

Simultaneously, the audiovisual
streaming subsystem managed to balance the trade-offs between media quality and
latency, providing an immersive, real-time gaming experience. Overall, the
system has demonstrated that with careful design and tuning, it is possible to
deliver remote gameplay that respects the original Wii experience.

Despite these successes, the project also encountered significant challenges and
limitations. One key drawback was the persistent latency observed throughout the
system. Although non-blocking I/O and aggressive buffering strategies reduced
delays, the cumulative effect of video capture, encoding, network transmission,
and emulation meant that some degree of lag remained -- particularly noticeable
in fast-paced gaming scenarios.

Additionally, limited peripheral support
constrains the current implementation; the focus on the Wii Remote excludes
important accessories like the nunchuck, thereby narrowing the scope of
supported games. Manual accelerometer calibration also posed a challenge, as the
fixed parameters did not adapt dynamically to different game requirements or
user preferences.

% \section{Future Work}

% \subsection{Enhanced Peripheral Integration}
% An avenue for future work involves expanding the system to support a broader
% range of Wii peripherals. For example, incorporating the nunchuck would not only
% enhance the gaming experience but also extend the system's compatibility with a
% wider variety of games that rely on additional input modalities. This expansion
% would require revisiting the current input relay architecture to integrate new
% input types and handling the unique sensor characteristics of each peripheral.
% Furthermore, the integration of these devices may necessitate revisions to the
% calibration routines and error-handling mechanisms, ensuring that the system
% processes the additional inputs with the same level of precision as the primary
% Wii Remote.

% \subsection{Scalability Testing}
% Future work should focus on scalability testing to evaluate how the data relay
% mechanisms perform when subjected to higher network loads. This entails setting
% up experiments that mimic real-world scenarios with multiple simultaneous
% connections, stressing the network to identify potential
% bottlenecks or points of failure in the project. By adopting robust simulation and testing
% methodologies, researchers could optimise the system’s architecture to better
% manage concurrent data streams and ensure that performance degrades gracefully
% under load. Ultimately, such testing will provide valuable insights into the
% system’s resilience and inform subsequent iterations of both software and
% hardware optimisations.

% \subsection{Latency Optimisation}
% Despite the progress made in reducing system latency, there is still room for
% improvement, especially for fast-paced gaming scenarios. Future work should
% explore advanced techniques for latency optimisation across all components of
% the system. This could involve a deeper investigation into the
% \texttt{WiimoteEmulator} fork and the input relay program to identify
% bottlenecks and streamline existing hot paths. Additionally, examining
% alternative RTP streaming parameters and exploring new network protocols might
% reveal opportunities to shave off critical milliseconds of delay.

% \subsection{Dynamic Calibration Techniques}
% Another important direction for future research is the development of dynamic
% calibration techniques for the accelerometer data. The current system relies on
% hand-tuned parameters, which, while effective in a controlled setting, may not
% perform optimally across different games or individual user preferences. Future
% work could focus on implementing adaptive algorithms that automatically adjust
% calibration settings in real time, using data gathered during gameplay. This
% might involve leveraging machine learning models to predict optimal calibration
% profiles based on user interactions or game-specific demands. By continuously
% refining the calibration process, the system can provide more accurate input
% replication, reducing discrepancies and further narrowing the performance gap
% between the emulated and original Wii experiences.

Looking ahead, several promising avenues exist for further refining and
expanding the system. Enhancing peripheral integration is a clear priority; by
incorporating additional input devices such as the nunchuck and other motion
sensors, the system could offer a more comprehensive emulation of the Wii’s
original capabilities. Furthermore, expanding the project to support multiple
simultaneous connections would fully recreate the local multiplayer experience,
allowing four players to engage in a game remotely. Future work might also
explore adaptive, dynamic calibration techniques -- potentially leveraging
machine learning -- to automatically fine-tune accelerometer inputs based on
real-time gameplay data.  Such advancements could narrow the performance gap
between emulated and native experiences, ultimately delivering a more authentic
and responsive gaming environment.

In conclusion, this dissertation has successfully explored the transformation of
a nostalgic gaming console into a modern, remote multiplayer platform.  While
certain technical challenges and limitations remain, the project functions as intended and allows for remote
multiplayer gameplay on the Nintendo Wii. The
insights gained from this work not only provide a foundation for future
innovations in remote gaming technology but also reaffirm the enduring appeal of
the classic Nintendo Wii experience.
