\chapter{Evaluation}
\label{chapter:evaluation}

This chapter evaluates the system with respect to the original objectives, and it critically compares the projects approach to related work in the field. Specifically, it assesses the system's playability, identifies challenges encountered during development, and outlines solutions to these issues. The chapter also discusses the limitations of the system and suggests areas for future work.

\section{Experimental Setup}
\label{sec:experimental-setup}

The core hardware comprised a Nintendo Wii system, two complete sets of Raspberry Pi units (each paired with a monitor and keyboard), and multiple Wii Remote controllers. A network switch was used to interconnect the devices on a closed, direct Ethernet network. To support gameplay evaluation, a disk copy of a split-screen multiplayer Wii game -- Mario Kart in this case -- was used, ensuring that the system was tested under conditions closely resembling a typical gaming scenario.

In the playability analysis setup, the configuration was modified slightly to accommodate specialised measurement requirements. The Wii’s video output was split using a composite splitter, which enabled the signal to be simultaneously routed to the USB composite capture device attached to one Raspberry Pi and to an extra monitor. This additional monitor served to display the original, real-time composite output from the Wii, providing a baseline for latency comparisons. This additional display allowed for a direct, real-time view of the Wii’s output which doesn't go through the capture device. By comparing the real-time composite feed with the processed stream, the analysis could isolate and quantify any additional delays introduced during capture, encoding, network transmission, and decoding.

\section{Playability Analysis}
\label{sec:playability-analysis}

The playability analysis takes into account 5 distinct parts of the system that
contribute to the overall interaction lag\cite{volkerseekerBestPaper}.

\subsection{Latency of Audiovisual Transmission}

The first test focused on quantifying the delay in the complete video path—from
the moment the signal left the Wii, through capture by the composite device,
encoding on the host machine, network transmission, decoding by the client
machine, and finally its display on the client monitor. To determine this delay,
both the client monitor and the additional monitor (showing the direct Wii
output) were recorded simultaneously. Analysis of the recordings revealed a
consistent lag of 21 frames between the two displays. Given that the video was
captured at 25 frames per second and played back at one-eighth speed, this delay
corresponds to approximately 105 milliseconds of lag. This measured latency
encompasses the cumulative delays inherent in video capture, encoding, and
decoding, offering a clear benchmark for the audiovisual subsystem’s
performance.

\subsection{Wii Remote Emulator Latency}

The second test evaluated the performance of the Wii Remote emulator itself. By
incorporating a \texttt{gettimeofday} call in the C code immediately upon receiving an
input from the network and again right after dispatching the corresponding
emulated input, the latency intrinsic to the emulation process was determined and output to the terminal.
The results, shown in Listing~\ref{lst:emulator}, indicate that the average
latency for IR data was 1511 microseconds (1.511 milliseconds), while the
accelerometer and button data exhibited average delays of 1498 and 5388
microseconds (1.498 and 5.388 milliseconds), respectively. These measurements
illustrate that the emulation system introduces only minimal delays.


\begin{lstlisting}[caption={Wii Remote Emulator Latency Results}, label=lst:emulator, mathescape=true,]
  IR:         average 1511 $\mu$s (1326 samples)
  Accelerometer: average 1498 $\mu$s (1457 samples)
  Button:     average 5388 $\mu$s (25 samples)
\end{lstlisting}

\subsection{Input Relay System Latency}

The third test assessed the efficiency of the Python-based input relay
system. Similar in concept to the emulator test, this measurement was conducted
by recording timestamps using Python’s \texttt{time.monotonic()} function immediately
before an input was processed and again right after it was sent over the UDP
socket then printing the results to the terminal. The results, shown in
Listing~\ref{lst:inputrelay}, show that the latency of the input relay system is
negligible, with average delays of 18, 13, and 18 microseconds for the
accelerometer, IR, and button inputs, respectively. These results confirm that the input
relay mechanism imposes an extremely minor delay.

\begin{lstlisting}[caption={Input Relay System Latency Results}, label=lst:inputrelay, mathescape=true]
INFO:wiimote_streamer:  Accelerometer: average 18 $\mu$s (3389 samples)
INFO:wiimote_streamer:  IR:            average 13 $\mu$s (3058 samples)
INFO:wiimote_streamer:  Button:        average 18 $\mu$s (26 samples)
\end{lstlisting}

\subsection{Network Latency}

The fourth test concentrated on the network latency between the host and client
machines. In this evaluation, the Raspberry Pis were connected directly via
Ethernet within a closed network, a configuration that minimises external
interference. The command \texttt{ping -c 100 -i 0.2 192.168.20.10} was executed to
collect latency statistics. The analysis, shown in Listing~\ref{lst:network},
revealed an average round-trip time (RTT) of 0.203 milliseconds as well as a
minimum and maximum RTT of 0.190 and 0.289 milliseconds, respectively. The results
demonstrate that the network latency of the system is very low. It is important to note that while these values demonstrate extremely low latency on a
dedicated network, typical home networks experience latencies in
the 30–100ms range on average\cite{latencySurvey}.

\begin{lstlisting}[caption={Network Latency Results}, label=lst:network]
--- 192.168.20.10 ping statistics ---
100 packets transmitted, 100 received, 0% packet loss, time 20195ms
rtt min/avg/max/mdev = 0.190/0.203/0.289/0.010 ms
\end{lstlisting}



\subsection{Bluetooth Latency:}

In this project, the Bluetooth latency of the Wii Remote was not directly
measured, however, measurements of Wii Remote latency have
been conducted by others. The MiSTer FPGA Input Latency
Tester\cite{cathoderayblogTestYour} is a tool that can measure the latency of
various controllers, including the Wii Remote. A collection of measurements from
this tool called the ``MiSTer Input Lag Database''\cite{misterInputLagDatabase}
shows that the Wii Remote has an average latency of 34.115 to 36.071
milliseconds. From this data, it can be inferred that the Bluetooth latency of
the Wii Remote is approximately 35 milliseconds which is a significant
contribution to the overall system latency.

\subsection{Overall Interaction Lag}

The interaction lag, defined as the delay between a user’s input and the
corresponding on-screen response\cite{volkerseekerBestPaper}, can be calculated by
summing the latencies of each part of the system. In this setup, the total interaction lag is:

\begin{align*}
  \text{Total Interaction Lag} &= \text{Audiovisual Latency} + \text{Wii Remote Emulator Latency} \\
  &+ \text{Input Relay System Latency} + \text{Network Latency} \\
  &+ (\text{Bluetooth Latency} * 2) \\
  &= 105 + 5.388 + 0.018 + 0.203 + (35 * 2) \\
  &= 180.609 \text{ milliseconds}
\end{align*}

This total is assuming the worst case (button press) for the Wii Remote emulator
and input relay system latencies. The Bluetooth latency is multiplied by two to
account for the fact that there is a Bluetooth connection between the Wii Remote
and the client machine as well as between the Wii Remote emulator running on the
host machine and the Wii Console. The network lag is only counted once as the
Audiovisual latency includes the network latency on the trip back to the client
machine. Note that using a typical home network with latencies in the 30-100ms
range would increase the total interaction lag to 240-380ms.

\section{Challenges and Solutions}

% \subsubsection{Input Relay and Data Synchronisation}
% Integrating the \texttt{xwiimote} library with the modified \texttt{WiimoteEmulator} fork presented challenges in synchronising accelerometer and IR data. Custom matrix transformations in \texttt{motion.c} and hand-tuned calibration routines ensured that the emulated signals closely replicated the physical Wii Remote behaviour. Additionally, the adoption of a binary protocol for transmitting sensor data reduced overhead and improved overall system responsiveness.

\subsubsection{IR Sensor Emulation}
At first, the IR emulation only mapped to the bottom half of the screen due to a scaling value error.

Originally, the vector for the three IR coordinates in the Wiimote emulators 3D space was as follows:
\begin{lstlisting}[style=CStyle, emph={vec3}, emphstyle={\color{magenta}}]
vec3 pointer_world = {(pointer_x - 0.5) * screen_width, (pointer_y - 0.5) * screen_width / screen_aspect, -screen_distance};
\end{lstlisting}

Changing removing the constant -0.5 and the screen aspect ratio from the y-coordinate calculation fixed the issue:
\begin{lstlisting}[style=CStyle, emph={vec3}, emphstyle={\color{magenta}}]
vec3 pointer_world = {(pointer_x - 0.5) * screen_width, (pointer_y)*screen_width, -screen_distance};
\end{lstlisting}

By correcting this error, the IR sensor data was correctly positioned on the screen, allowing for accurate pointing and cursor control. This fix was crucial for maintaining the playability of IR-dependent games.

\subsubsection{Audiovisual Streaming}
The initial implementation of the streaming pipeline suffered from high latency and frequent buffering issues, resulting in a sub-optimal user experience. To address these challenges, the system underwent several rounds of optimisation, focusing on reducing delay while maintaining media quality.

Manual tests refined the audiovisual streaming component by evaluating each parameter adjustment for its impact on latency and video quality. After trying each option, the system’s response was observed in real time to assess whether the perceptible latency met the requirements for responsive gameplay and whether any degradation in video quality was within acceptable limits.

Initially, the adoption of options such as \texttt{-preset ultrafast} and \linebreak \texttt{-tune zerolatency} resulted in the most significant reduction in noticeable latency. However, these settings also led to a decline in video quality, introducing compression artefacts and reduced image clarity that hampered the overall viewing experience. In contrast, incorporating the \texttt{-crf 17} parameter improved the visual quality substantially. Although it introduced a slight increase in the latency, this trade-off proved acceptable as the added delay was barely perceptible, while the quality enhancement ensured that details remained crisp and legible during fast-paced gameplay.

This iterative testing and fine-tuning process ultimately converged on the final set of streaming parameters, shown in section~\ref{sec:audio-video-streaming}, that best balanced low latency with high video quality. The final configuration represents a carefully considered compromise, ensuring that the audiovisual streaming pipeline delivers a responsive and visually satisfactory experience for remote gameplay.


\subsubsection{Latency}

Latency is a central performance metric in this project, as it directly impacts
both the responsiveness of audiovisual feedback and the accuracy of input relays
during gameplay. The overall system latency encompasses several stages: video
capture, encoding, network transmission, decoding, and input processing. Each of
these stages has been the focus of targeted optimisation efforts.

On the streaming side, the low latency results from a combination of
high-speed encoding and aggressive buffering control. The use of \texttt{ffmpeg}
with options like \texttt{-g:v 1} (forcing every frame as a keyframe) helps in
rapid recovery and minimises delays during packet loss or re-synchronisation
events. The broadcasting pipeline’s non-buffered approach, as enforced by the
\texttt{-max\_delay 0} and \texttt{-fflags +nobuffer} flags, keeps the
end-to-end delay to a minimum, although this makes the system more
sensitive to network jitter.

The input relay component, responsible for transmitting controller data from the
Wii Remote to the emulator, further mitigates latency by employing non-blocking
I/O and a lightweight binary protocol. By packaging sensor data (IR and
accelerometer values) into fixed-length packets and transmitting them over UDP,
the system avoids the overhead associated with more complex data formats.

Despite these efforts, some latency remains when compared to native Wii
gameplay. As shown in \fullref{sec:playability-analysis}, the system
has a total interaction lag of approximately 200milliseconds with the audiovisual
streaming contributing 105 milliseconds of this. This latency is noticeable in
fast-paced games but is still within a playable range.

\section{Limitations}

Although the system meets its primary objectives, several limitations remain that constrain its overall performance and usability. In this section, each limitation is discussed in detail, accompanied by potential solutions to address them in future iterations.

\subsection{Peripheral Support}

One significant limitation of the current implementation is its restricted
support for peripheral devices. At present, the system only handles input from
the primary Wii Remote, excluding peripherals such as the nunchuck -- a crucial peripheral for many Wii
titles. This limitation not only narrows the scope of emulated experiences but
also diminishes the authenticity of gameplay, as many games rely on the
complementary inputs of both controllers. The absence of nunchuck support
restricts the system’s appeal and usability, especially for titles that require
coordinated two-handed control. Addressing this limitation would involve extending
the existing input relay architecture to include additional routines for
capturing and processing nunchuck signals.

\subsection{Scalability}

Another key limitation is the system’s scalability. During testing, the setup
was evaluated with only a single remote player, leaving its performance under
multi-user scenarios unexamined. This single-user focus raises concerns about
how the system will manage increased network load, concurrent input streams, and
potential synchronisation issues when multiple players connect simultaneously.
Scalability issues may manifest as network congestion, increased latency, or
even data loss in real-world multi-player environments. To resolve these
challenges, future research should incorporate scalability testing under
simulated conditions that mimic the demands of multiple concurrent connections.
Moreover, potential improvements might include optimising the network
communication protocols, adopting load balancing techniques, or restructuring
the input relay to efficiently handle the increased volume of simultaneous
inputs without compromising performance.

\subsection{Latency}

Latency remains a critical limitation, as it directly impacts both the
responsiveness of the audiovisual streaming and the precision of input relays.
Despite several optimisation strategies -- such as the use of non-blocking I/O,
tuned streaming settings, and emulation improvements -- a noticeable delay
persists when compared to native Wii gameplay. This residual latency arises from
the cumulative delay introduced at various stages, including video capture,
encoding, network transmission, decoding, input processing, and emulation.
However, the largest contributor to latency is the audiovisual streaming
pipeline, which accounts for approximately 105 milliseconds of the total
interaction lag as shown in section~\ref{sec:playability-analysis}. To further
mitigate latency, future work could explore more advanced compression
algorithms, refine the UDP transmission pipeline, or experiment with alternative
low-latency streaming protocols. Additionally, implementing predictive buffering
or adaptive latency compensation algorithms might help offset some of the
unavoidable delays, thereby enhancing the overall responsiveness of the system.

\subsection{Accelerometer Calibration}

The final limitation concerns the calibration of the accelerometer emulation.
Currently, the system relies on hand-tuned parameters to interpret accelerometer
data, which may not be universally optimal across all games or user preferences.
This manual calibration approach can lead to inconsistencies, where the
sensitivity and accuracy of motion controls vary between different games. In
some cases, this might necessitate game-specific adjustments to achieve
satisfactory performance, thereby undermining the system’s general
applicability. A promising solution to this problem is the use of dynamic
calibration techniques that automatically adjust the accelerometer settings in
real time. Using these techniques, the system could analyse gameplay data on the
fly and fine-tune the calibration parameters to match the current game’s
requirements and individual user behaviour. Additionally, incorporating a
user-friendly calibration interface would empower users to make personalised
adjustments, further bridging the gap between the emulated inputs and the
original Wii experience. Furthermore, if the Wii Remote emulator was able to
perfectly replicate the accelerometer data then there would be no need for
calibration which would be the ideal solution.

\section{Reflection}

\subsection{Evaluation of Objectives}
As stated in \fullref{chapter:introduction}, the key objectives of the project were:

\begin{enumerate}
	\item  Develop a system to capture and stream the Wii’s video and audio output to remote players.
	\item Develop a system to relay the Wii Remote’s controller data over a low-latency network connection.
	\item Evaluate the system’s performance and user experience in a real-world setting.
\end{enumerate}

Reflecting on the project in respect to these objectives, it is clear that the
project successfully fulfils all three objectives. The system developed is
capable of capturing and streaming the Wii’s video and audio output to remote
players, relaying the Wii Remote’s controller data over a low-latency network
connection, and has been evaluated in a real-world setting. Section~\ref{sec:playability-analysis} demonstrates that the system is capable of
providing a playable experience for users, despite some limitations.

% \subsection{Evaluation of Objectives}
% Reflecting on the project’s primary goals, it is evident that the system has successfully addressed the key challenges outlined in the \hyperlink{chapter:introduction}{Introduction}.

% \subsubsection{Objective 1: Capturing and Streaming the Wii's Video and Audio Output}
% This objective focuses on the development and integration of automated scripts
% specifically designed to capture and stream the Wii's audiovisual data. The
% broadcasting script (\texttt{broadcast-rtp.sh}) plays a crucial role by
% leveraging the Real-time Transport Protocol (RTP) in conjunction with
% \texttt{ffmpeg} to encode and transmit video and audio data with minimal
% buffering. The script's carefully tuned parameters -- including non-buffered input,
% forced keyframes, and low-latency encoding settings ensure that the streaming
% pipeline efficiently manages the trade-off between media quality and
% transmission delay. On the client side, the playback script
% (\texttt{play-rtp.sh}) mirrors these optimisations to maintain synchronisation
% and reduce decoding overhead. Together, these scripts automate the processes
% required for capturing the Wii’s output, ensuring that remote players receive a
% real-time and immersive multimedia experience.

% \subsubsection{Objective 2: Relaying the Wii Remote's Controller Data over a Low-Latency Network Connection}
% The second objective centres on the relay of controller data from the Wii Remote
% to the Wiimote emulator. First, a custom Python-based input relay system, built
% using the xwiimote Python bindings, captures real-time input events -- including
% button presses, accelerometer data, and IR signals -- and translates them into
% fixed-length binary packets. By adopting a lightweight binary protocol over UDP,
% the input relay minimises overhead, thereby ensuring low-latency transmission.
% In addition, enhancements made to the Wiimote Emulator enable it to accurately
% interpret these packets, thereby seamlessly integrating the physical controller
% inputs into the virtual environment. The combined approach of precise input
% relay and robust Wii Remote emulation confirms that the system meets the
% objective of providing responsive and accurate control for remote gameplay.

% \subsubsection{Objective 3: Evaluating System Performance and User Experience}
% The final objective involves a comprehensive playability analysis measuring
% system performance under realistic conditions. This analysis quantifies latency
% across different system stages, with detailed measurements assessing network
% latency during data transmission, the overhead introduced by the emulation
% process, and the interaction lag\cite{volkerseekerBestPaper} -- that is, the delay between a user’s input
% and the corresponding on-screen response. The findings indicate that while the
% latency is within a playable range, a measurable interaction lag remains. This
% evaluation not only validates the system's capability to deliver a playable
% gaming experience but also highlights specific areas where further optimisation
% can bring the performance closer to that of native Wii gameplay.


\subsection{Comparison with Related Work}

\subsubsection{WiimoteEmulator}
The original \texttt{WiimoteEmulator} project by rnconrad and subsequent forks
(e.g., JRogaishio's version) primarily focused on emulating the Wii Remote for
local control using Bluetooth. In contrast, this work extends these foundations
by implementing IR and accelerometer emulation over IP sockets. This system adapts the concept of
Wii Remote emulation to enable remote gameplay -- a feature not present in the
original projects.

\subsubsection{Audiovisual Streaming in Remote Gaming}
Cloud gaming platforms traditionally deploy extensive infrastructure and
proprietary solutions to manage audiovisual streams. This allows for
high-quality, low-latency streaming, but it comes at huge monetary and technical
cost\cite{cloudSurvey}.
In contrast, this project's solution uses open-source tools and
protocols to achieve similar functionality. This system has higher latency and lower quality than
commercial cloud gaming services, but everything can run locally for free on low-end hardware.

\subsubsection{Wii Online Services}
The original Wii Online services offered a better user experience than this
project, as its design was tailor made for the Wii's hardware and software.
However, Nintendo discontinued these services in
2014\cite{nintendoTerminationNintendo}. Some third-party projects like Wiimmfi\cite{wiimmfi}
have focused on replicating the functionality of the Wii’s original online
services. However, many Nintendo Wii games have specific gameplay features that
require local multiplayer or never supported online play -- such as Wii Sports.
This project aims to address this limitation by enabling remote multiplayer
experiences for games that were not designed for online play.
