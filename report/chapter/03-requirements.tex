\mychapter{3}{Requirements Specification}
\label{chapter:requirements}

\section{Functional Requirements}

\subsubsection{Video and Audio Capture and Streaming:}
	      The system shall capture the Wii’s video and audio outputs and stream them to remote players with minimal latency. This functionality is critical to preserve the fluid, immersive experience typical of classic Wii titles.

\subsubsection{Controller Input Relay:}
	      The solution must reliably capture and transmit Wii Remote inputs—including motion data and button presses—over a low-latency network connection. This bi-directional communication is essential for maintaining the real-time responsiveness expected in interactive gameplay.

\subsubsection{Synchronization:}
	      To ensure a seamless gaming experience, audiovisual data and controller inputs must be synchronized. The system should adjust for network variability and maintain precise timing to replicate local multiplayer dynamics.




\section{Non-Functional Requirements}

\subsubsection{Performance:}
	      The system must operate under strict low-latency conditions to minimize delay and jitter. Efficient processing and optimized data streaming protocols are required.

\subsubsection{Reliability and Robustness:}
	      The solution should tolerate variations in network quality, ensuring continuous, stable operation even under less-than-ideal conditions.

\subsubsection{Usability:}
	      An intuitive interface and straightforward setup process should be provided, enabling users to connect and enjoy games with minimal technical intervention.

\subsubsection{Evaluation:}
	      Comprehensive testing in real-world environments is necessary. Both quantitative performance metrics and qualitative user feedback will be gathered to assess the overall experience.


