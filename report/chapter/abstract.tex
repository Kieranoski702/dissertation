\chapter*{Abstract}

The Nintendo Wii is well-known for its innovative, motion-based controls and engaging, family-friendly games such as Mario Kart Wii. Despite its hardware limitations compared to modern consoles, its local multiplayer experiences have cultivated a devoted following. However, with the rapid shift toward online gaming, recreating the Wii’s in-person, split-screen experiences has become increasingly challenging. This project proposes a solution that vitalises the Wii’s input and output interfaces, enabling remote players to enjoy an experience that mirrors local multiplayer gaming.

The approach centres on a three key components. First, video and audio streaming techniques capture the Wii’s outputs and deliver them to remote devices using low-latency protocols. Second, a novel controller input relay system transmits Wii Remote signals, including motion and button inputs, over a network. Third, a Wii Remote emulator interprets the remote player’s inputs and forwards them to the Wii console.

By combining these components, the project enables remote players to participate in local multiplayer games on the Wii. Furthermore, it establishes a framework for adapting retro systems to contemporary, distributed gaming environments. The work not only preserves the social and communal essence of local play but also offers broader implications for making nostalgic gaming experiences accessible to players across geographically separated locations.

\begin{center}
	\noindent\rule{8cm}{0.4pt}
\end{center}
