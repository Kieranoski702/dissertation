\chapter{User Manual}
\label{appendix:user-manual}

Before proceeding with any operations, ensure that you have run the \texttt{setup.sh} script. This script installs and compiles all necessary dependencies, so there is no need to build or compile components manually.

\section{Determining Bluetooth Addresses}

\subsection{Determining the Wii Remote Address}
\label{subsection:wii-remote-address}

\begin{enumerate}
	\item Use the command:
	      \begin{verbatim}
hcitool scan
    \end{verbatim}
	      to search for discoverable devices.
	\item Press the red sync button (located under the battery cover) on the Wii remote.
	\item The Wii remote should appear (usually listed as \texttt{Nintendo RVL-CNT-01}) with its corresponding MAC address.
\end{enumerate}

\subsection{Determining the Wii Console Address}
\label{subsection:wii-console-address}

If the Wii remote emulator is operational, simply press the console's sync button. The emulator will automatically connect and display the Wii console's address.

\section{Wii Remote Connection to Client Machine}

To connect a Wii remote to the client machine, follow these steps:

\begin{enumerate}
	\item \textbf{Identify the Wii Remote's MAC Address:}
	      Follow the steps outlined in Section \ref{subsection:wii-remote-address} to determine the Wii remote's MAC address.

	\item \textbf{Establish a Connection:}
	      \begin{enumerate}
		      \item Open a terminal and launch \texttt{bluetoothctl}.
		      \item Enable scanning by typing
		            \begin{verbatim}
scan on
\end{verbatim}
		      \item Type the command
		            \begin{verbatim}
trust [address]
        \end{verbatim}
		            while pressing the 1 and 2 buttons on the Wii remote which will put the Wii Remote into sync mode.
		      \item Next, type
		            \begin{verbatim}
connect [address]
        \end{verbatim}
		            ensuring that the Wii Remote is still in sync mode by pressing the 1 and 2 buttons.
	      \end{enumerate}
\end{enumerate}

% For further details, please refer to the online guide available at \url{https://blog.malware.re/2023/07/04/Wiimote-on-Linux-with-dev-input/index.html}.

\section{Running the Wii Remote Emulator and Input Relay}

To emulate Wii remote functionality, perform the following steps from within the \texttt{WiimoteEmulator} directory:

\begin{enumerate}
	\item \textbf{Disable and Stop the Bluetooth Service:}
	      \begin{verbatim}
sudo systemctl disable bluetooth
sudo systemctl stop bluetooth
sudo systemctl status bluetooth
    \end{verbatim}

	\item \textbf{Start the Custom Bluetooth Daemon Manually:}
	      \begin{verbatim}
sudo ./bluez-4.101/dist/sbin/bluetoothd
    \end{verbatim}

	\item \textbf{Launch the Emulator:}
	      Run the emulator with the Wii Console's MAC address (as determined in Section \ref{subsection:wii-console-address})
	      and specify the IP address and port:
	      \begin{verbatim}
sudo ./wmemulator [MAC Address] [IP Address] [Port]
    \end{verbatim}

	\item \textbf{Start the Python Input Relay on client machine:}
	      On another machine, run the input relay script:
	      \begin{verbatim}
./input_relay.py --host [IP Address] --port [Port]
\end{verbatim}


	\item \textbf{Restoring Bluetooth Services:}
	      Once finished, execute the following on the host machine:
	      \begin{verbatim}
sudo killall bluetoothd
sudo systemctl enable bluetooth
sudo systemctl start bluetooth
sudo systemctl status bluetooth
    \end{verbatim}
\end{enumerate}

\section{Streaming Wii Video to Another Machine}

To stream video from the Wii, set up the host and client machines as follows:

\begin{enumerate}
	\item \textbf{On the Host Machine:}
	      Run the broadcasting script:
	      \begin{verbatim}
./broadcast-rtp.sh
    \end{verbatim}

	\item \textbf{On the Client Machine:}
	      Run the playback script:
	      \begin{verbatim}
./play-rtp.sh
    \end{verbatim}


	\item \textbf{Terminate the Streaming Session:}
	      When you wish to stop streaming, press \texttt{q} on the client machine and then terminate the broadcast process by pressing \texttt{Ctrl+C} on the host machine.
\end{enumerate}
