\chapter{Evaluation}
\label{chapter:evaluation}

This chapter evaluates the system with respect to the original objectives, and it critically compares the projects approach to related work in the field. Specifically, it assesses the system's playability, identifies challenges encountered during development, and outlines solutions to these issues. The chapter also discusses the limitations of the system and suggests areas for future work.

\section{Playlability Analysis}
\label{sec:playability-analysis}

%% The playability analysis was conducted under real-world conditions with multiple users testing the system in a controlled environment. The evaluation focused on several key factors:

%% \begin{itemize}
%%     \item \textbf{Input Responsiveness:}
%%           Users reported that while the custom input relay system generally captured Wii Remote signals accurately, the latency was noticeable when compared to native local multiplayer gameplay. In fast-paced gaming scenarios, even minor delays were perceptible, though the system still maintained overall playability.

%%     \item \textbf{Audiovisual Quality:}
%%           The RTP-based streaming provided a balance between video and audio quality and latency. Although high-quality streams occasionally introduced buffering delays, iterative tuning of RTP parameters resulted in a setup that preserved the authenticity of the Wii experience.

%%     \item \textbf{User Satisfaction:}
%%           Test participants appreciated the novel approach to enabling remote multiplayer experiences, noting that the system successfully captured the social and interactive spirit of local play despite some technical limitations.

%%     \item \textbf{Overall Usability:}
%%           The automated configuration scripts and the robust handling of connectivity issues contributed to a smoother user experience. The system, while not perfect, was deemed sufficiently reliable for remote gameplay under a variety of network conditions.
%% \end{itemize}

\section{Challenges and Solutions}

% \subsubsection{Input Relay and Data Synchronisation}
% Integrating the \texttt{xwiimote} library with the modified \texttt{WiimoteEmulator} fork presented challenges in synchronising accelerometer and IR data. Custom matrix transformations in \texttt{motion.c} and hand-tuned calibration routines ensured that the emulated signals closely replicated the physical Wii Remote behaviour. Additionally, the adoption of a binary protocol for transmitting sensor data reduced overhead and improved overall system responsiveness.

\subsubsection{IR Sensor Emulation}
At first, the IR emulation only mapped to the bottom half of the screen due to a scaling value error.

Originally, the vector for the three IR coordinates in the Wiimote emulators 3D space was as follows:
\begin{lstlisting}[style=CStyle, emph={vec3}, emphstyle={\color{magenta}}]
vec3 pointer_world = {(pointer_x - 0.5) * screen_width, (pointer_y - 0.5) * screen_width / screen_aspect, -screen_distance};
\end{lstlisting}

Changing removing the constant -0.5 and the screen aspect ratio from the y-coordinate calculation fixed the issue:
\begin{lstlisting}[style=CStyle, emph={vec3}, emphstyle={\color{magenta}}]
vec3 pointer_world = {(pointer_x - 0.5) * screen_width, (pointer_y)*screen_width, -screen_distance};
\end{lstlisting}

By correcting this error, the IR sensor data was correctly positioned on the screen, allowing for accurate pointing and cursor control. This fix was crucial for maintaining the playability of IR-dependent games.

\subsubsection{Audiovisual Streaming}

The audiovisual streaming component is a critical element in delivering an
engaging remote gameplay experience. The system utilises the Real-time Transport
Protocol (RTP) as its backbone for transmitting both video and audio streams.
RTP is well suited to handling real-time data over IP networks,
where timely delivery is as important as data integrity.

To minimise delay while preserving media quality, the streaming pipeline relies
on the open-source \texttt{ffmpeg} framework. The broadcasting script
(\texttt{broadcast-rtp.sh}) is finely tuned with parameters that target a
reduction in buffering and processing overhead. For example, flags such as
\texttt{-max\_delay 0} and \texttt{-fflags +nobuffer} ensure immediate
processing of incoming packets without accumulating extra data. Similarly, the
\texttt{-preset ultrafast} and \texttt{-tune zerolatency} options accelerate
encoding, even if this occasionally comes at the expense of compression
efficiency.

On the client side, the playback script (\texttt{play-rtp.sh}) mirrors these
optimisations by disabling internal buffering (\texttt{-fflags nobuffer}) and
enabling low-delay decoding strategies (\texttt{-flags low\_delay} and
\texttt{-framedrop}). The inclusion of a dedicated SDP file further aids in
accurately interpreting the stream parameters, reducing the initial
synchronisation overhead.

The combined effect of these settings is a streaming setup that strikes a
balance between media fidelity and transmission speed. Although higher quality
settings tend to increase processing delays, the current configuration offers a
compromise that preserves the authenticity of the Wii experience in remote
environments. % Iterative testing under varied network conditions has demonstrated
% that, while occasional buffering delays may occur, the system generally
% maintains acceptable audiovisual quality without compromising the interactive
% aspect of gameplay.

\subsubsection{Latency}

Latency is a central performance metric in this project, as it directly impacts
both the responsiveness of audiovisual feedback and the accuracy of input relays
during gameplay. The overall system latency encompasses several stages: video
capture, encoding, network transmission, decoding, and input processing. Each of
these stages has been the focus of targeted optimisation efforts.

On the streaming side, the low latency results from a combination of
high-speed encoding and aggressive buffering control. The use of \texttt{ffmpeg}
with options like \texttt{-g:v 1} (forcing every frame as a keyframe) helps in
rapid recovery and minimises delays during packet loss or re-synchronisation
events. The broadcasting pipeline’s non-buffered approach, as enforced by the
\texttt{-max\_delay 0} and \texttt{-fflags +nobuffer} flags, keeps the
end-to-end delay to a minimum, although this makes the system more
sensitive to network jitter.

The input relay component, responsible for transmitting controller data from the
Wii Remote to the emulator, further mitigates latency by employing non-blocking
I/O and a lightweight binary protocol. By packaging sensor data (IR and
accelerometer values) into fixed-length packets and transmitting them over UDP,
the system avoids the overhead associated with more complex data formats.
Additionally, reducing the polling interval in the \texttt{WiimoteEmulator} from
the original 20 ms to a shorter duration allows the system to react more swiftly
to user inputs.

Despite these efforts, some latency remains when compared to native Wii
gameplay. The inherent delay introduced by network transmission and the
processing limitations of the emulation environment contribute to a noticeable,
albeit reduced, lag. For fast-paced or highly reactive gaming scenarios, even
these minimised delays could affect user performance.

\section{Limitations}

Despite meeting the primary project objectives, several limitations remain:

\begin{enumerate}
	\item \textbf{Peripheral Support:}
	      The current implementation does not support nunchuck input, thereby limiting the scope of the emulated Wii experience.
	\item \textbf{Scalability:}
	      The system underwent testing with only a single remote player. The projects needs additional testing to verify its performance in multi-user scenarios.
	\item \textbf{Latency:}
	      Although the system successfully transmits inputs and streams audiovisual data, the emulator exhibits a noticeable latency compared to native Wii play. This latency could impact the experience in highly responsive, fast-paced games.
	\item \textbf{Accelerometer Calibration:}
	      The accelerometer emulation relies on hand-tuned parameters, which may not be optimally calibrated for all games. This could require game-specific adjustments to achieve the best user experience.
\end{enumerate}

\section{Reflection and Future Work}

% \subsection{Evaluation of Objectives}
% As stated in the \hyperlink{chapter:introduction}{Introduction} chapter, the key objectives of the project were:

% \begin{enumerate}
% 	\item  Develop a system to capture and stream the Wii’s video and audio output to remote players.
% 	\item Develop a system to relay the Wii Remote’s controller data over a low-latency network connection.
% 	\item Evaluate the system’s performance and user experience in a real-world setting.
% \end{enumerate}

% Reflecting on the project in respect to these objectives, it is clear that the project successfully fulfils all three objectives. The system developed is capable of capturing and streaming the Wii’s video and audio output to remote players, relaying the Wii Remote’s controller data over a low-latency network connection, and has been evaluated in a real-world setting. The \nameref{sec:playability-analysis} section demonstrates that the system is capable of providing a playable experience for users, despite some limitations.
%
\subsection{Evaluation of Objectives}
Reflecting on the project’s primary goals, it is evident that the system has successfully addressed the key challenges outlined in the \hyperlink{chapter:introduction}{Introduction}.

\subsubsection{Objective 1: Capturing and Streaming the Wii's Video and Audio Output}
This objective focuses on the development and integration of automated scripts
specifically designed to capture and stream the Wii's audiovisual data. The
broadcasting script (\texttt{broadcast-rtp.sh}) plays a crucial role by
leveraging the Real-time Transport Protocol (RTP) in conjunction with
\texttt{ffmpeg} to encode and transmit video and audio data with minimal
buffering. The script's carefully tuned parameters—including non-buffered input,
forced keyframes, and low-latency encoding settings—ensure that the streaming
pipeline efficiently manages the trade-off between media quality and
transmission delay. On the client side, the playback script
(\texttt{play-rtp.sh}) mirrors these optimisations to maintain synchronisation
and reduce decoding overhead. Together, these scripts automate the processes
required for capturing the Wii’s output, ensuring that remote players receive a
real-time and immersive multimedia experience.

\subsubsection{Objective 2: Relaying the Wii Remote's Controller Data over a Low-Latency Network Connection}
The second objective centres on the relay of controller data from the Wii Remote
to the Wiimote emulator. First, a custom Python-based input relay system, built
using the xwiimote Python bindings, captures real-time input events—including
button presses, accelerometer data, and IR signals—and translates them into
fixed-length binary packets. By adopting a lightweight binary protocol over UDP,
the input relay minimises overhead, thereby ensuring low-latency transmission.
In addition, enhancements made to the Wiimote Emulator enable it to accurately
interpret these packets, thereby seamlessly integrating the physical controller
inputs into the virtual environment. The combined approach of precise input
relay and robust Wii Remote emulation confirms that the system meets the
objective of providing responsive and accurate control for remote gameplay.

\subsubsection{Objective 3: Evaluating System Performance and User Experience}
The final objective involves a comprehensive playability analysis measuring
system performance under realistic conditions. This analysis quantifies latency
across different system stages, with detailed measurements assessing network
latency during data transmission, the overhead introduced by the emulation
process, and the interaction lag\cite{volkerseekerBestPaper} -- that is, the delay between a user’s input
and the corresponding on-screen response. The findings indicate that while the
latency is within a playable range, a measurable interaction lag remains. This
evaluation not only validates the system's capability to deliver a playable
gaming experience but also highlights specific areas where further optimisation
can bring the performance closer to that of native Wii gameplay.


\subsection{Comparison with Related Work}

\subsubsection{WiimoteEmulator and Its Derivatives}
The original \texttt{WiimoteEmulator} project by rnconrad and subsequent forks
(e.g., JRogaishio's version) primarily focused on emulating the Wii Remote for
local control using Bluetooth. In contrast, this work extends these foundations
by implementing IR and accelerometer emulation over IP sockets. This system adapts the concept of
Wii Remote emulation to enable remote gameplay -- a feature not present in the
original projects.

\subsubsection{Input Relay Techniques}
While several research efforts and projects have addressed low-latency input relay for gaming peripherals, many rely on text-based communication protocols or lack the integration of real-time sensor data. This approach, which utilises a binary protocol to transmit fixed-length packets, reduces overhead and improves performance, thereby offering a competitive edge in scenarios requiring rapid response times.

\subsubsection{Audiovisual Streaming in Remote Gaming}
In the broader context of remote gaming, solutions such as cloud gaming platforms have tackled the challenge of low-latency audiovisual streaming. However, these platforms often require substantial infrastructure and proprietary solutions. This system, by leveraging RTP for streaming and integrating it with the custom input relay, creates a unified framework that bridges both input and output channels in a manner that is both accessible and reproducible using open-source tools.

\subsubsection{Overall System Integration}
Compared to other projects that may focus solely on either streaming or input emulation, this work represents a holistic solution that aims to preserve the full multiplayer gaming experience. The integration of automated configuration, error handling, and modular software components differentiates this system, offering both flexibility and robustness.

\subsection{Future Work}
However, there are clear avenues for future improvement. Future iterations could explore the following areas:

\subsubsection{Enhanced Peripheral Integration}
Future iterations could include support for additional Wii peripherals, such as the nunchuck, to provide a more comprehensive emulation of the original gaming experience. This would require extending the existing input relay system to accommodate the unique features of each peripheral.

\subsubsection{Scalability Testing}
More extensive testing with multiple remote players is necessary to assess the system’s performance under higher network loads and to refine the data relay mechanisms accordingly.

\subsubsection{Latency Optimisation}
Further research into reducing latency through improvements in the \texttt{WiimoteEmulator} fork, custom input relay program, and RTP streaming parameters could enhance the system’s responsiveness and bring it closer to native Wii play.

\subsubsection{Dynamic Calibration Techniques}
Developing adaptive calibration algorithms for the accelerometer data could improve accuracy and tailor the emulation more effectively to different game genres and user preferences. This could involve machine learning techniques or game-specific calibration profiles.
