\chapter{Introduction}
\hypertarget{chapter:introduction}{}

This project addresses a pressing challenge in the evolution of gaming experiences: how to adapt and extend the social and immersive qualities of local multiplayer systems -- exemplified by the Nintendo Wii -- to a modern, online environment. The Nintendo Wii has gained a large and dedicated following largely due to its motion controls, low price, and its local multiplayer gameplay. However, as online gaming has become the norm, the traditional split-screen and communal experiences that defined the Wii era have faced diminishing support and new technical challenges.

The primary aim of this project is to bridge the gap between classic local multiplayer gameplay and the demands of modern distributed gaming. This is achieved by re-engineering both the output and input interfaces of the Wii. On one hand, the project focuses on capturing and streaming audio and video from the console using low-latency protocols to preserve the fluidity and authenticity of the original experience. On the other hand, a novel controller input relay system has been developed to transmit Wii Remote signals -- including motion data, IR readings, and button presses -- over a network to remote devices. In doing so, the system tackles challenges inherent in Bluetooth communication, network variability, and the need for precise synchronisation between audiovisual streams and control inputs.

The key objectives of the project were:
\begin{enumerate}
\item  Develop a system to capture and stream the Wii’s video and audio output to remote players.
\item Develop a system to relay the Wii Remote’s controller data over a low-latency network connection.
\item Evaluate the system’s performance and user experience in a real-world setting.
\end{enumerate}

Throughout this report, the reader will find detailed discussions of the technical design, implementation challenges, and testing procedures that collectively contribute to a solution aimed at revitalising retro gaming experiences. The subsequent chapters present an in-depth analysis of the system architecture, innovative design decisions, and the experimental validation of the proposed solution.


\begin{center}
	\noindent\rule{8cm}{0.4pt}
\end{center}



