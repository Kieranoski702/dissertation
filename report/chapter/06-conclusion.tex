\chapter{Conclusion}
\label{chapter:conclusion}

This dissertation set out to explore the feasibility of remote multiplayer
gameplay on the Nintendo Wii console while retaining local multiplayer dynamics.
Through the development of a comprehensive system that integrates audiovisual
streaming, low-latency input relay, and Wii Remote emulation, the project has
successfully demonstrated the feasibility of remote gameplay that closely mimics
the native local multiplayer experience.

The project’s most significant achievement lies in its successful integration of
multiple complex components into a cohesive system. The implementation of a
custom input relay, in tandem with a modified Wii Remote emulator, allowed for
real-time transmission of controller inputs. This solution ensured that remote
players could experience accurate and responsive control, even though some
latency remained compared to native gameplay. Simultaneously, the audiovisual
streaming subsystem managed to balance the trade-offs between media quality and
latency, providing an immersive, real-time gaming experience. Overall, the
system has demonstrated that with careful design and tuning, it is possible to
deliver remote gameplay that respects the original Wii experience.

Despite these successes, the project also encountered significant challenges and
limitations. One key drawback was the persistent latency observed throughout the
system. Although non-blocking I/O and aggressive buffering strategies reduced
delays, the cumulative effect of video capture, encoding, network transmission,
and emulation meant that some degree of lag remained -- particularly noticeable
in fast-paced gaming scenarios. Additionally, limited peripheral support
constrains the current implementation; the focus on the Wii Remote excludes
important accessories like the nunchuck, thereby narrowing the scope of
supported games. Manual accelerometer calibration also posed a challenge, as the
fixed parameters did not adapt dynamically to different game requirements or
user preferences.

Looking ahead, several promising avenues exist for further refining and
expanding the system. Enhancing peripheral integration is a clear priority; by
incorporating additional input devices such as the nunchuck and other motion
sensors, the system could offer a more comprehensive emulation of the Wii’s
original capabilities. Furthermore, scalability testing and further optimisation
are essential for accommodating multiple simultaneous connections without
sacrificing performance. Future work might also explore adaptive, dynamic
calibration techniques -- potentially leveraging machine learning -- to
automatically fine-tune accelerometer inputs based on real-time gameplay data.
Such advancements could narrow the performance gap between emulated and native
experiences, ultimately delivering a more authentic and responsive gaming
environment.

In conclusion, this dissertation has successfully explored the transformation of
a nostalgic gaming console into a modern, remote multiplayer platform.  While
certain technical challenges and limitations remain, the achievements outlined
herein demonstrate the potential for further innovation in this space. The
insights gained from this work not only provide a foundation for future
innovations in remote gaming technology but also reaffirm the enduring appeal of
the classic Nintendo Wii experience.
