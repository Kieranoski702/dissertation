\chapter{Context Survey}
\label{chapter:context}

This section surveys the broader context of the project by reviewing the historical background, key technologies, and recent initiatives that align with the aim of vitalising local multiplayer experiences. In particular, it examines the Nintendo Wii’s ecosystem, the evolution of its input devices, and the supporting technologies that have enabled both commercial and experimental adaptations.

\section{The Nintendo Wii and Its Ecosystem}
Released by Nintendo in 2006, the Wii quickly became renowned for its innovative motion-based controls and engaging titles. Central to its appeal was the Wii Remote (Wiimote), a wireless controller equipped with accelerometers, infrared sensors, and traditional button inputs. These features enabled intuitive, physical interactions, helping to bridge the gap between digital gameplay and physical movement. Over time, the Wii’s local multiplayer format -- often characterised by split-screen or shared-screen experiences -- cemented its legacy as a console that prioritised communal play.

\section{Relevant Hardware and Software Technologies}
Modern adaptations of the Wii experience leverage a range of hardware and software tools:

\subsubsection{\texttt{WiimoteEmulator}\cite{wiimote_emulator}:} This publicly available project on GitHub allows for the emulation of Wii Remote signals, enabling a real Wii console to interface with a computer acting as an external controller. By emulating the communication protocol of the Wiimote, the project provides a basis for further experimentation with input methods. In the context of this dissertation, a fork of the \texttt{WiimoteEmulator} has been extended to accept IR and accelerometer data from across a network. This extension is key to bridging remote inputs with local emulation.

\subsubsection{\texttt{xwiimote} Library\cite{xwiimote}:} To capture real Wiimote input, the \texttt{xwiimote} library has been employed. Running on a Raspberry Pi, this library facilitates the interfacing of physical Wiimote hardware with software, thereby enabling the capture and processing of motion and button data. This data is then routed through a custom Python script that integrates with the extended emulation system, ensuring that remote control signals are correctly interpreted.

\subsubsection{Raspberry Pi:} The Raspberry Pi serves as a versatile, low-cost computing platform that supports the integration of various peripherals and communication protocols. In this project, the Raspberry Pi is used to capture Wiimote data from a client machine and relay it to the emulation system on the host machine which interfaces with the Nintendo Wii console.


\section{Recent Work and Similar Endeavours}

The landscape of remote gaming and controller emulation is relatively niche, with few projects addressing the dual challenge of low-latency audiovisual streaming and precise controller input relay. Beyond the core WiimoteEmulator project, the following points are noteworthy:

\subsubsection{Controller Emulation for Legacy Consoles:} Prior research has largely focused on the emulation of input devices for legacy consoles in order to preserve or extend their operational lifespan. Such projects have typically emphasised local connectivity and hardware replication. The extension to network-based control -- wherein sensor data such as IR and accelerometer signals are transmitted remotely -- is less common and represents a novel contribution of this work.

\subsubsection{Remote Gaming Frameworks:} In recent years, there has been increased interest in remote gaming solutions, driven by advancements in streaming protocols and low-latency communication. While many contemporary projects target high-end gaming platforms, the retro gaming sphere has seen fewer contributions that successfully bridge the gap between traditional, hardware-based control schemes and modern, networked gameplay.

% TODO: Maybe cut this
%% \subsubsection{Tool and Technology Integration:} The use of open-source libraries such as \texttt{xwiimote} alongside custom software modifications to existing projects (e.g., the \texttt{WiimoteEmulator} fork) illustrates a growing trend in leveraging community-driven tools to solve complex emulation challenges. Although a comprehensive body of literature specific to this integration is still emerging, the available work provides a solid foundation for exploring how retro systems can be adapted for contemporary, distributed gaming environments.

%% \section{Broader Implications and Potential Directions}

%% By synthesising advancements in motion-based control, emulation technologies, and network streaming, this project contributes to a deeper understanding of how traditional gaming experiences can be adapted to meet modern demands. The approach not only revitalises a beloved console experience but also establishes a framework that can be extended to other legacy systems. Furthermore, the integration of real-time controller data with audiovisual streams presents new opportunities for research in synchronisation, network resilience, and user experience design in distributed environments.

